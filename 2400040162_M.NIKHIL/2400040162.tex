\documentclass[12pt,a4paper]{article}

% Essential packages
\usepackage[utf8]{inputenc}
\usepackage[margin=1in]{geometry}
\usepackage{graphicx}
\usepackage{xcolor}
\usepackage{listings}
\usepackage{fancyhdr}
\usepackage{titlesec}
\usepackage{enumitem}
\usepackage{hyperref}
\hypersetup{
    hidelinks
}

% ==== THEME COLORS ====
\definecolor{themeBlue}{HTML}{0047AB}
\definecolor{lightGray}{HTML}{F5F5F5}

% ==== FONT CHANGE TO MODERN SANS-SERIF ====
\renewcommand{\familydefault}{\sfdefault}
\usepackage{helvet}

% ==== CODE BLOCK THEME ====
\lstset{
    basicstyle=\ttfamily\small,
    breaklines=true,
    frame=single,
    numbers=left,
    numberstyle=\tiny,
    backgroundcolor=\color{lightGray},
    keywordstyle=\color{themeBlue}\bfseries
}

% ==== HEADER/FOOTER THEME ====
\pagestyle{fancy}
\fancyhf{}
\rhead{\textcolor{themeBlue}{Open Source Report}}
\lhead{\textcolor{themeBlue}{Mandava Nikhil}}
\rfoot{\textcolor{themeBlue}{Page \thepage}}
\renewcommand{\headrulewidth}{1pt}
\renewcommand{\headrule}{\hbox to\headwidth{\color{themeBlue}\leaders\hrule height \headrulewidth\hfill}}

% ==== SECTION STYLE ====
\titleformat{\section}
  {\Large\bfseries\color{themeBlue}}
  {\thesection}{1em}{}

\titleformat{\subsection}
  {\large\bfseries\color{black}}
  {\thesubsection}{1em}{}

\titleformat{\subsubsection}
  {\normalsize\bfseries\color{themeBlue}}
  {\thesubsubsection}{1em}{}

\begin{document}

% ============ FRONT PAGE ============
\begin{titlepage}
    \centering
    \vspace*{1cm}
    
    {\Huge\bfseries\color{themeBlue} Open Source Software Report\\[0.5cm] \par}
    
    \vspace{2cm}
    
    {\Large\bfseries Student Details\par}
    \vspace{1cm}
    
    \begin{tabular}{ll}
        \textbf{Name:} & MANDAVA NIKHIL \\[0.3cm]
        \textbf{Roll Number:} & 2400040162 \\[0.3cm]
        \textbf{Department:} & ELECTRONICS AND COMMUNICATION ENGINEERING \\[0.3cm]
        \textbf{University:} & KL University \\[0.3cm]
        \textbf{Course:} & Open Source \\[0.3cm]
        \textbf{Semester:} & ODD SEMESTER\\[0.3cm]
    \end{tabular}
    
    \vfill
    
    {\large Submitted to:\\[0.3cm]
    \textbf{Dr. Sripath Roy}\\
    Department of Computer Science and Engineering\\
    KL University\par}
    
    \vspace{1cm}
\end{titlepage}

\newpage
\tableofcontents
\newpage

% ================================
% 1. LINUX DISTRO
% ================================
\section{Linux Distribution}

\subsection{Distribution Used: Ubuntu 22.04 LTS}

Ubuntu 22.04 LTS is a widely used Linux operating system that provides Long Term Support, stability, and security updates for 5 years, making it ideal for development and server hosting.

\subsection{Why Ubuntu?}

\begin{itemize}
    \item \textbf{User-Friendly}: Easy to install and use
    \item \textbf{Stable and Secure}: Long-term support
    \item \textbf{Developer Tools}: Supports Git, Node.js, Python, GCC
    \item \textbf{Server Ready}: Commonly used for cloud deployments
\end{itemize}

\subsection{System Specifications}

\begin{itemize}
    \item OS: Ubuntu 22.04 LTS
    \item Kernel: Linux 5.15
    \item Architecture: x86\_64
    \item Shell: Bash 5.1
\end{itemize}

\newpage

% ================================
% 2. ENCRYPTION AND GPG
% ================================
\section{Encryption and GPG}

\subsection{What is Encryption?}

Encryption converts readable data into ciphertext so that only authorized users with the correct key can read it.

\subsection{Types of Encryption}

\subsubsection{Symmetric Encryption}
\begin{itemize}
    \item Same key for encryption and decryption
    \item Fast and efficient
    \item Examples: AES, DES
\end{itemize}

\subsubsection{Asymmetric Encryption}
\begin{itemize}
    \item Uses a public and private key
    \item Commonly used for secure communication
    \item Example: RSA
\end{itemize}

\subsection{What is GPG?}

GPG (GNU Privacy Guard) is an open-source encryption tool used to:
\begin{itemize}
    \item Encrypt data
    \item Decrypt messages
    \item Digitally sign documents
\end{itemize}

\subsection{Installing GPG}
\begin{lstlisting}[language=bash]
sudo apt update
sudo apt install gnupg
\end{lstlisting}

\subsection{Generating Keys}

\begin{lstlisting}[language=bash]
gpg --full-generate-key
\end{lstlisting}

\newpage

% ================================
% 3. ENCRYPTED EMAIL
% ================================
\section{Sending Encrypted Email}

\subsection{Why Encrypt Email?}

Emails can be intercepted across networks. Encryption ensures privacy, integrity, and authenticity.

\subsection{Tool Used: Mozilla Thunderbird}

Thunderbird supports OpenPGP encryption for secure communication.

\subsection{Steps to Send Encrypted Email}

\begin{enumerate}
    \item Install Thunderbird
    \item Add email account
    \item Import GPG keys
    \item Enable encryption before sending
\end{enumerate}

\newpage

% ================================
% 4. PRIVACY TOOLS
% ================================
\section{Privacy Tools from Prism-Break.org}

\subsection{1. Signal}

\begin{itemize}
    \item End-to-end encrypted messaging
    \item Open source and secure
\end{itemize}

\subsection{2. Tor Browser}

\begin{itemize}
    \item Anonymous browsing
    \item Protects against tracking and surveillance
\end{itemize}

\subsection{3. Tutanota}

\begin{itemize}
    \item Encrypted email service
    \item Open-source clients
\end{itemize}

\subsection{4. KeePassXC}

\begin{itemize}
    \item Password manager
    \item Stores data locally in encrypted format
\end{itemize}

\subsection{5. OnionShare}

\begin{itemize}
    \item Share files anonymously over Tor
    \item No central server required
\end{itemize}

\newpage

% ================================
% 5. OPEN SOURCE LICENSE
% ================================
\section{Open Source License}

\subsection{License Used: MIT License}

The MIT License is a permissive open-source license that allows anyone to use, copy, modify, and distribute the software as long as the original license is included.

\begin{lstlisting}
MIT License
Copyright (c) 2025
Permission is hereby granted...
\end{lstlisting}

\newpage

% ================================
% 6. SELF-HOSTED SERVER
% ================================
\section{Self-Hosted Server: PairDrop}

\subsection{What is PairDrop?}

PairDrop is a peer-to-peer file sharing tool that allows devices on the same network to transfer files directly using WebRTC, without internet or user accounts.

\subsection{Why I Selected PairDrop}

\begin{itemize}
    \item Lightweight and fast
    \item No signup or cloud storage
    \item Perfect for LAN file sharing
\end{itemize}

\subsection{Installation on Ubuntu}

\textbf{Step 1: Install Node.js and npm}
\begin{lstlisting}[language=bash]
sudo apt update
sudo apt install nodejs npm -y
\end{lstlisting}

\textbf{Step 2: Clone Repository}
\begin{lstlisting}[language=bash]
git clone https://github.com/schlagmichdoch/PairDrop.git
cd PairDrop
\end{lstlisting}

\textbf{Step 3: Install Dependencies}
\begin{lstlisting}[language=bash]
npm install
\end{lstlisting}

\textbf{Step 4: Start Server}
\begin{lstlisting}[language=bash]
npm start
\end{lstlisting}

\subsection{Accessing PairDrop}

\begin{lstlisting}
http://localhost:3000
http://<local-ip>:3000
\end{lstlisting}

\subsection{PairDrop Web Interface}

\begin{figure}[h]
    \centering
    \includegraphics[width=0.95\linewidth]{pairdrop_web.png}
    \caption{PairDrop Web Interface}
\end{figure}

\subsection{Self-Hosted Demonstration at KL University}

\begin{figure}[h]
    \centering
    \includegraphics[width=0.95\linewidth]{pairdrop_demo.jpg}
    \caption{PairDrop Demonstration at KL University}
\end{figure}

\subsection{Localization (Telugu Translation)}

To make PairDrop accessible to Telugu users, I translated key documentation sections.

\subsubsection{Translated Title}

\textbf{PairDrop ఫైల్ షేరింగ్ డాక్యుమెంటేషన్}

\subsubsection{Translated Terms}

\begin{itemize}
    \item Installation – ఇన్‌స్టాలేషన్
    \item Server – సర్వర్
    \item Network – నెట్‌వర్క్
    \item File Transfer – ఫైల్ ట్రాన్స్‌ఫర్
\end{itemize}

\subsection{Poster Overview}
GitHub

GitHub - schlagmichdoch/PairDrop: PairDrop: Transfer Files Cross-Platform. No Setup, No Signup.

PairDrop: Transfer Files Cross-Platform. No Setup, No Signup. - schlagmichdoch/PairDrop
A poster was designed to showcase:
\begin{itemize}
    \item What is PairDrop
    \item How it works
    \item Key features
    \item Benefits of self-hosting
\end{itemize}

\newpage

% ================================
% 7. OPEN SOURCE CONTRIBUTIONS
% ================================
\section{Open Source Contributions (PRs)}

\subsection{Repositories Contributed}

\begin{enumerate}
    \item \textbf{fineanmol/hacktoberfest} – Added quick guide
    \item \textbf{yfosp/start-here} – Added name and profile
    \item \textbf{zero-to-mastery/start-here-guidelines} – Contributor update
    \item \textbf{firstcontributions/first-contributions} – Added name
\end{enumerate}

\subsection{Summary}

\begin{itemize}
    \item Total PRs: 4
    \item Merged: 4
    \item Open: 0
\end{itemize}

\newpage

% ================================
% 8. LINKEDIN POSTS
% ================================
\section{LinkedIn Posts}

\subsection{Post 1: First Open Source Contribution}
\url{https://www.linkedin.com/feed/update/urn:li:activity:7399151134055055360}

\subsection{Post 2: Self-Hosting PairDrop}
\url{https://www.linkedin.com/in/mahi-korrapati-a24773369/}

\subsection{Post 3: Open Source Journey}
\url{https://www.linkedin.com/feed/update/urn:li:activity:7399163090891587584}

\newpage

% ================================
% 9. CONCLUSION
% ================================
\section{Conclusion}

This report helped me explore:
\begin{itemize}
    \item Linux and command-line usage
    \item GPG and secure communication
    \item Privacy tools
    \item Open-source licensing
    \item Hosting a real application
    \item GitHub collaboration
\end{itemize}

It strengthened my understanding of open-source culture, security, and real-world deployment.

\end{document}
Linkedin

Just completed my first open-source contribution on GitHub! Really happy about this achievement. https://lnkd.in/dvz5ehvF | Mandava Nikhil

